\section{Event Handling}
Reaktive Systeme sind per definition asynchron, dH sie treten zu einem beliebigen Zeitpunkt ein.

\subsection{C-Implementation}
\textbf{main.h}
\begin{lstlisting}
#include "framework.h"
	
void onCallback(int param) {
	printf("Callback executed");
}
	
int main(void) {
	registerCallback(onCallback);
	run();
}
\end{lstlisting}

\textbf{framework.h}
\begin{lstlisting}
typdef void (*Callback)(int);

static Callback notify;

void registerCallback(Callback c) {
	notify = c;
}

void run() {
	if (notify != 0) {
		notify(4);
	}
}
\end{lstlisting}

\subsection{C++ Implementation I}
\textbf{main.h}
\begin{lstlisting}
#include "framework.h"

class Part {
public: 
	void onCallback(int param) {
		printf("Callback executed");
	}
};

int main(void) {
	Part p;
	Framework f;
	
	f.registerCallback(p, &Part::onCallback);
	f.run();
}
\end{lstlisting}

\textbf{framework.h}
\begin{lstlisting}

class Part; // Class Forward-Declaration

class Framework {
public:
	void registerCallback(Part& p, void (Part::*pFct)(int)) {
		part = &p;
		callback = pFct;
	}
	
	void run() {
		(part->*callback)(5);
	}

private:
	Part* part;
	void (Part::*callback)(int);
};
\end{lstlisting}

\subsection{C++ Implementation II}
\textbf{main.h}
\begin{lstlisting}
#include "framework.h"

class Part : public IPart {
public: 
	virtual void onCallback(int param) override {
		printf("Callback executed");
	}
};

int main(void) {
	Part p;
	Framework f;
	
	f.registerCallback(p);
	f.run();
}
\end{lstlisting}

\textbf{framework.h}
\begin{lstlisting}

class IPart {
public:
	virtual void onCallback(int param) = 0;
	virtual ~IPart() {};	
};

class Framework {
	public:
	void registerCallback(Part& p) {
		part = &p;
	}
	
	void run() {
		if (part != nullptr) {
			part->onCallback(5);
		}
	}
	
	private:
	IPart* part;
};
\end{lstlisting}