\section{Projektplanung}
\subsection{Plangesteuert}
In der klassischen Projektplanung werden folgende Schritte durchgeführt:\\
\begin{tabular}{lm{5.5cm}}
	\textbf{Strukturplan} & Der Strukturplan legt fest, wie ein Projekt organisiert bzw. welches Vorgehen gewählt wird. Das Projekt wird mithilfe der definierten Anforderungen in \textbf{Arbeitspakete} unterteilt (In Scrum könnten das zB User Stories sein). \\
	\textbf{Aufwandschätzung} & Einzelene Arbeitspakete (UserStories) werden abgeschätzt.\\
	\textbf{Arbeitsplan} & Die voneinander abhängigen Arbeitspakete werden vernetzt und oft auch an Entwickler zugewiesen. (zB mit Gantt-Diagramm) \\
	\textbf{Kostenplanung} & Kosten werden abgeschätzt \\
	\textbf{Risikomanagement} & Risiken bewertet und allfällig behoben
\end{tabular}

\subsection{Agil}
In der moderneren Projektplanung, bei der nach einem Agilen Prinzip gearbeitet wird, sind folgende Schritte durchzuführen:
\begin{tabular}{lm{5.5cm}}
	\textbf{Releaseplanung} & Welche Epics (Gruppierung von UserStories) werden für welche Version geplant. \\
	\textbf{Iterationsplanung} & Welche UserStories müssen in der nächste Iteration umgesetzt werden \\
	\textbf{Aufwandschätzung} & UserStories müssen abgeschätzt werden \\
	\textbf{Aufgabenplanung} & UserStories in Tasks unterteilen. zB für UserStore: \textit{Tempo Messen} mehrere Task umgesetzt werden (Sensor auslesen, Filtern, Darstellen usw)
\end{tabular}

\subsection{Aufwandschätzung}
Um Aufwände zu schätzen muss Anforderung bzw Pflichtenheft so detailiert wie möglich sein. Arten zur Schätzung:
\begin{itemize}[nosep]
	\item Bauchgefühl
	\item Algorithmisch - Eine Formel suchen und pro Task anwenden
	\item Vergelichen - Anhand abgeschlossener Projekte vergleiche ziehen
	\item Exportenbefragung - Know-How von Mitarbeitern nutzen
\end{itemize}

In der klassischen Projektplanung werden Stunden pro Task geschätzt. Für agile Methoden können auch StoryPoints pro UserStory vergeben werden. Ziel ist es, eine Stundenunabhängige Einheit zu schaffen (Dies löst das Problem, dass ein Task von Mitarbeiter A nicht gleichschnell gelöst wird wie von Mitarbeiter B). Jeder Mitarbeiter weiss, wieviele StoryPoints er pro Itartion erledigen kann.